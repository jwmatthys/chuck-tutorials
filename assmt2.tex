\documentclass{article}
\title{ChucK Assignment 2}
\date{5 April 2012}
%\usepackage{fullpage}
\usepackage{listings}
\usepackage{url}

\begin{document}
\maketitle

\textsl{In this lesson you will learn about variables, conditional tests, and how to perform repetitive tasks automatically. You'll also explore the properties of the natural harmonic series.}\vspace{2mm}

\section{Variables}
\begin{enumerate}
\item We're going to continue to use oscillators for this lesson. Review the
techniques from lesson 1 for connecting and controlling oscillators.

\item Sometimes we need to be able to store and recall numbers, times, and
other pieces of information. This is what \textbf{variables} are for. Do you remember
variables from high school algebra? Well, forget them! The concept of
variables in computers is completely different.

\begin{itemize}
\item Variables are just \textsl{containers.} Each container holds one thing. That
thing might be a number, or a duration, or a piece of text, or even an oscillator.

\item In order to use a variable, you first must create it, give it a name, and
indicate what kind of information it will hold.

\item The three most common variable types are \textbf{int}, \textbf{float},
and \textbf{dur}, representing integers, floating-point numbers, and durations.
We'll discuss more about the differences between these later.
\end{itemize}

Here is the code to create a new integer variable called \textbf{myNum}:

\begin{lstlisting}
	int myNum;
\end{lstlisting}

Now we can store a value in that variable:

\begin{lstlisting}
	25 => myNum;
\end{lstlisting}

That variable can now be used in place of a simple number in your code:

\begin{lstlisting}
	TriOsc myOsc => dac;
	myNum => myOsc.freq;
\end{lstlisting}

\item Here's the cool twist: variables can be changed, and you can use the
\textsl{old} value of a variable to set the \textsl{new} value.

\begin{lstlisting}
	int myFrequency;
	55 => myFrequency;
	<<< myFrequency >>>; // This prints the value on screen
	myFrequency + 100 => myFrequency;
	<<< myFrequency >>>; // Now it's 155
\end{lstlisting}

\end {enumerate}

\section {Conditional Tests}
\begin {enumerate}

\item Now that we have numbers in variables, we sometimes want to use them
to make decisions. For this we use the \textbf{if} command:

\begin{lstlisting}
float myGain;
0.8 => myGain;
if (myGain > 1)
{
  <<< "myGain is greater than 1">>>;
} else
{
  <<< "myGain is not greater than 1">>>;
}
\end{lstlisting}

\item The \textbf{conditional test} is the part inside the parentheses. In this case, myGain is 0.8, so the test \textsl {is myGain greater than 1?} is 
false.

\item The following are possible conditional tests:

\begin{lstlisting}
x > y		is x greater than y?
x < y		is x less than y?
x >= y		is x greater than or equal to y?
x <= y		is x less than or equal to y?
x == y		does x equal y? (notice 2 equal signs)
x != y		is x not equal to y?
\end{lstlisting}

\end {enumerate}

\section {Repetition}

\begin{enumerate}
\item If you want to do something many times, you could just copy and paste
it in your code, but that's not a very efficient way of working. There are
several methods of doing repeated things in ChucK. The first is the
\textbf{repeat} command.

\begin{lstlisting}
repeat(10)
{
  <<<"Hi, nice to meet you! I have no short term memory!">>>;
}
\end{lstlisting}

\item We can use repeat to create a series or pattern by changing a variable
in every repetition.

\begin{lstlisting}
int myNumber;
5 => myNumber;
repeat(10)
{
  <<< myNumber >>>;
  myNumber + 1 => myNumber; // increase it by 1
}
\end{lstlisting}

\textsl{There is a useful shortcut for increasing a number by 1: myNumber++}

\item You are probably familiar with the natural harmonic series. It is 
produced by playing frequencies that are all multiples of the same fundamental.

\begin{lstlisting}
TriOsc myOsc => dac;
0.5 => myOsc.gain;
int fundamental;
int harmonic;
55 => fundamental;
1 => harmonic;
repeat(10)
{
  fundamental * harmonic => myOsc.freq; // multiply
  <<< "frequency:",fundamental * harmonic >>>; // print it
  harmonic++; // increase by 1
  250::ms => now;
}
\end{lstlisting}

\textsl{EXTRA CREDIT: There are multiple methods of creating the
harmonic series. Rewrite the above code to do the same thing but using
a different method of calculating the frequency.}

\item More complicated than \textbf{repeat}, but much more flexible, is
the \textbf{for} loop. A \textbf{for} loop contains a conditional test,
so you can decide whether you want to keep looping or not.

A \textbf{for} loop looks like this:

\begin{lstlisting}
for(int num; num < 10; num++)
{
  <<< num >>>;
}
\end{lstlisting}
\pagebreak
See the three different parts in the parentheses?

\begin{lstlisting}
       1         2       3
for(int num; num < 10; num++)
\end{lstlisting}


\begin{itemize}
\item The first part is the \textsl{initialization.} This is something
done just before it begins looping. In this case, a new \textbf{int} called
num is created.
\item The second part is the \textsl{conditional test}. The commands in the
brackets will be followed it it's true. If it's false, the looping ends.
\item The third part is the \textsl{enumeration}. Simply put, it's what
happens at the end of every repetition.
\end{itemize}

\textbf{You can remember the form of a for loop with the mnemonic ICE:}
\begin{lstlisting}
I: Initialization
C: Conditional test
E: End of Every repetition
\end{lstlisting}

\item Here is the same harmonic series code as above, but now using
a \textbf{for} loop instead of \textbf{repeat}:

\begin{lstlisting}
TriOsc myOsc => dac;
0.5 => myOsc.gain;
int fundamental;
55 => fundamental;
for (1 => int harmonic; harmonic < 11; harmonic++ )
{
  fundamental * harmonic => myOsc.freq;
  250::ms => now;
}
\end{lstlisting}
\end{enumerate}

\textbf{Assignment 2: Harmonic Series Etude}\vspace{2mm}
\\
\textbf{Experiment with the harmonic series code above. Using variables
and a repeat() or for() loop, create a short etude that explores the
harmonic series. Be creative: try to go beyond the simple idea of an
ascending arpeggio. Save your ChucK code as a .ck or .txt file and
upload it to the assignment page on Blackboard by this Tuesday, April 10.}

\end{document}
